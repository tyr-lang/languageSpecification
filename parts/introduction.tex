
\section{Introduction}

Type-oriented programming (TOP) is a paradigm that states that types are objects.
In consequence, it is possible to perform calculations on types like any other calculation.
As it is true for objects in object-oriented programming (OOP), types can be copied and may have mutable state.
The mutable state of a type can be bounded by static knowledge in the same way as pointer can be restricted to point to objects of a certain type.
As such, TOP implies OOP.

Tyr is a programming language created to explore this idea in practice.
Tyr as a language is a descendant of C++ and Scala.
In order to examine consequences of TOP for resource management, Tyr features manual memory management.

Type-level functions are descendants of C++ templates and Ada generics.


\subsection{On OOP and TOP}

TOP has no point without OOP for the following reason:

\begin{lstlisting}
type A;
type B;

val x = new (if(phi) A else B)();
\end{lstlisting}

What could be the type of x if not $A \sqcap B$?
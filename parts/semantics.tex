\section{Semantics}

The semantics of Tyr is loosely based on C++ and Scala.

\subsection{Sorts}

Types in Tyr are split into five sorts: properties, interfaces, value types, types and classes.
The sort of properties is independent of all other sorts.
Interfaces are the only secondary sort.
All remaining sorts are primary sorts.

The sort of value types, i.e. type originating from \verb+type+ declarations, forms the top of the hierarchy.
Types originating from \verb+var type+ and classes can inherit from their own sort or from value types.

An interface may inherit from any primary sort.

If a class does not explicitly inherit from a primary sort, it implicitly inherits from \texttt{tyr.lang.Object}.
If a non-class does not explicitly inherit from a primary sort, it implicitly inherits from \texttt{tyr.lang.any}.

Members from primary and secondary sorts may inherit from an arbitrary amount of interfaces, as long as all transitively inherited primary sorts form a legal line of inheritance.
The lowest type in that line is the actual super type, even if it is more specific than the declared super type.

\subparagraph{Note} A warning shall be issued if the actual super type is more specific than an explicitly declared super type.



%draw picture:
% properties
%
% var type -> type <- class
%  A           A         A
%  |           |         |
%  +------ interface ----+


\subsection{Types of literal Values}
\todo{das stimmt so nicht mehr; die aktuelle Regel ist aber imo besser}
\todo{man sollte erwähnen, dass 0i32 <: int gilt}

The type of <Integer> is \texttt{int}, if no 'i' or no number behind the 'i' is supplied.
If a number is supplied, the number will be used as argument for \texttt{Integer}(n).
If the resulting type has a named subtype in \texttt{tyr.lang}, the subtype will be chosen.
Hence, \texttt{0i8} is a \texttt{byte} of value 0.
Also, \texttt{0}, \texttt{0i} and \texttt{0i32} are indistinguishable.

The type of <HexInteger> is UnsignedInteger(n), where n is the number supplied via 'i' defaulting to 32.

The type of <long> is \texttt{long}.

The type of <Float> is analogous to <Integer> except that it is based on \texttt{FloatingPoint}(n) and defaults to \texttt{double}.
The type is \texttt{float} if a single \texttt{f} is supplied.
This rule is designed to be compatible with common programming languages.

The type of <string> is \texttt{LiteralString}.

<Identifier> literals yield identifiers.
An identifier is neither a type nor a value.

\subsection{Unescaping of String and Identifier Literals}

Tyr uses the same escaping mechanism as Java.
Unescaping happens for <string> and <identifier> before further processing.


\subsection{Access Paths introduced by local \texttt{with}}

Any experession that yields a scope can be imported with a with.
This import results in an access path to the names imported into the local scope.
If that access has a side effect, the effect may be executed at every access to the scope.
Function with an implicit \texttt{this} parameter start with an implicit expression "\texttt{with this;}".
The last part of an import expression must be an identifier that is used to create a name in the scope that the import occurs in.

\begin{lstlisting}[language=tyr]
with tyr.lang; // legal, makes name lang accessible
with lang.int; // legal, makes tyr.lang.int accessible under name int
with 7.toString(); // illegal, last part is a method
with 7.toString().bytes; // legal, will make the bytes of the string "7" accessebile under the name bytes; note: the call will be reevaluated on each usage
with 7.type; // legal, makes tyr.lang.LiteralInt available unter the name type
with "".type; // illegal, because the name type is already used in this scope (previous import)
with "".class; // legal
\end{lstlisting}


\subsection{Scopes}

As a consequence of assigning a type to any semantic entity, any entity is a scope.

Scope access happens via the dot-Operator ('.').
The static parent relationship of scopes forms a tree.
The root is called \texttt{\_root\_}.

If a scope is imported via a \texttt{with}-clause, imported scopes are addressed before ascending into parents.
If an imported scope is evaluated, parent scopes will be followed until a scope is encountered that does not belong to a user defined entity, i.e. that is a pure namespace.
This rule ensures that importing a type makes operations from its super types visible but not from its enclosing scope.

Including inheritance, scopes form a DAG.
Imports are not transitive.
Therefore, the set of visible entities can be enumerated using recursive descent on the scope DAG.

\todo{if an import ends with an \texttt{\_all\_} then all members of the target entity will be imported}

\paragraph{Dynamic Lookup} The first lookup strategy is called dynamic lookup and is always applied first.
It searches the scope of the target entity.
The scope of a value is its static type.
The scope and its imports are queried.
If no result can be found, the super types/classes are queried recursively.
If no results can be found, the super interfaces are queried.
\todo{specify that super types are linearized in topological order to prevent distant interfaces shadowing closer interfaces; note to self: in order to keep memory complexity of this feature linear, we have to assign numbers in a single pass before looking at expressions; evluation is performed by a heap containing all unevaluated super types; the key in the heap is said number}


\paragraph{Static Lookup} If dynamic lookup failed, a static lookup is performed.
This lookup follows the scope tree, including imports done by the scope.
If this lookup fails at \texttt{\_root\_}, a single flat lookup in \texttt{tyr.lang} is performed.


\subsubsection{Namespaces}

There is only one namespace for all kinds of named entities.
In a lookup, fields count like parameterless methods.


\subsubsection{Overloading and Overriding}

Find a good rule for near and perfect matches and function application.
Note to self: The first call following a member access has to be evaluated before the member access!

There is a function lookup mode that applies if the member access is followed by a call or in the case of operator usage.
This mode yields a set of results. \todo{rules?}
The evaluation of a member access in a simple expression yields a distance ordered non-empty sequence of results.
Parameterless entities are at the first position.
Only one parameterless entity will be presented, as it will shadow any other entity anyway.
If the user of the member access expects a single entry, the first element of that sequence is chosen.
Otherwise a matching element can be picked.

\todo{E-O-E-Regel gegen unary operators}

\todo{Single result? Implicit conversions?}

\begin{lstlisting}[language=tyr]
- 7 == -7;
int.`-`(7) == -7;
int.`-` 7 == -7;  // name resolution error?
\end{lstlisting}

\subsection{Asignment Operator}

Every type implicitly has a right-associative operator \texttt{=} with precedence 20 that implements assignment.
The argument type can be of any subtype of this.type and the result type and value is taken from the assignment, because it is more specific.
Hence, it is possible to do the following:
\begin{lstlisting}[language=tyr]
var x : int = 0
var y : literalInteger[32] = 0
y = x = 1
\end{lstlisting}

If an applicable operator \texttt{=} is visible, the implicit assignment operator is shadowed ignoring precedence.

\todo{Assignment to a val in a Constructor results in an FBOL and cannot be nested.}

\subsection{InstanceOf Operator}

Every type implicitly has a left-associative operator \texttt{<:} with precedence 90 that implements dynamic\footnote{The result may be evaluated at compile-time.} type-checking.
The argument type can be any type and the result is \texttt{bool}.
The result is true iff the dynamic type of the value of the left argument is a equal to or a subtype of the dynamic type represented by the right argument.

\todo{a warning shall be generated if the result is always false}

Hence, it is possible to do the following:
\begin{lstlisting}[language=tyr]
var x : int = 0
var y = x <: 0.type // y == false, as !(int <: literalInteger(32))
var z = literalInteger(32) <: int // z == false
var z = literalInteger(32) <: int.type // z == true
\end{lstlisting}


\subsection{Representation of Types}

A type application is the representation of a type.
It can be equivalent to a direct usage of a type instance; in that case, the source of the instance is the source type application is the representative of the equivalence class.


\subsection{Types and Classes}

\subsubsection{Representation of Objects}

Rule: only conversions and instances of types with known representation are legal, i.e. there cannot be a variable/result/parameter of type Any.
Note to self: There is a backdoor using type variables?
What about p = block : Block[Any]?

\subsubsection{Static and Dynamic Type of an Expression}

The static type of an expression can be accessed via the field \texttt{type}.
The dynamic type of an expression can be accessed via the field \texttt{class}.
The dynamic type is only available for Objects.
In a type safe program, the statement $\forall x. x\texttt{.class} <: x\texttt{.type}$ is true.
For the sake of symmetry, expressions argument to an access of \texttt{type} will be evaluated.


\subsubsection{Representation of Classes}

Classes, i.e. the objects representing the type of an instance of a class type, have an inexpressible polymorphic representation.
The first element of an object is a pointer to its type.
A class is an object, hence the rule applies to classes as well.

The second element of a Class is a pointer to the ITable.

Representations of classes and types are a co-Hierarchy to their defined counterparts.
Members with \texttt{class} lifetime behave like their object counterparts except that they target the class.

Both object and class defs translate to function pointer value fields in the class.
If a def is overriden, the overriding constructor replaces the value assigned by a super constructor rather than adding a new field.

Classes are created by the package initializer in arbitrary topological order.
Class constructors may not make use of non-native types other than their super types and interfaces.

Note: As soon as there is type-sealing, optimizing non-dispatching members to global entities should be legalized.


\subsubsection{Overriding of Fields}

Overriding of fields is covariant, as access via get to an overriden field requires dynamic type checking, which cannot be inserted for already compiled super types (hence no contra variance).

A practical example of field overriding are pointers between co-hierarchies such as object-type descriptions.
An object description points to a type description, whereas a function descriptions has this pointer pointing to a function type description.
The same holds for declaration-definition co-hierarchies.


\subsection{Tests}

Tests are methods in a class that have no surrounding instance, i.e. static methods.
They have no name and will not be exported to other modules or to runnable programms.
Tests can violate arbitrary visibility rules to simplify test code.
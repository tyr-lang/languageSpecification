\section{Semantics}

The semantics of Tyr is loosely based on C++ and Scala.

\subsection{Types of literal Values}
\todo{sind die ints nicht in wahrheit LiteralInt?}

The type of <Integer> is \texttt{int}, if no 'i' or no number behind the 'i' is supplied.
If a number is supplied, the number will be used as argument for \texttt{Integer}(n).
If the resulting Type has a named subtype in \texttt{tyr.lang}, the subtype will be chosen.
Hence, \texttt{0i8} is a \texttt{byte} of value 0.
Also, \texttt{0}, \texttt{0i} and \texttt{0i32} are indistinguishable.

The type of <HexInteger> is UnsignedInteger(n), where n is the number supplied via 'i' defaulting to 32.

The type of <long> is \texttt{long}.

The type of <Float> is analogous to <Integer> except that it is based on \texttt{FloatingPoint}(n) and defaults to \texttt{double}.
The type is \texttt{float} if a single \texttt{f} is supplied.
This rule is designed to be compatible with common programming languages.

The type of <string> is \texttt{LiteralString}.

<Identifier> literals yield identifiers.
An identifier is neither a type nor a value.

\subsection{Unescaping of String and Identifier Literals}

Tyr uses the same escaping mechanism as Java.
Unescaping happens for <string> and <identifier> before further processing.


\subsection{Access Paths introduced by local \texttt{with}}

Any experession that yields a scope can be imported with a with.
This import results in an access path to the names imported into the local scope.
If that access has a side effect, the effect may be executed at every access to the scope.
Function with an implicit \texttt{this} parameter start with an implicit \texttt{with this;} expression.


\subsection{Types and Classes}

\subsubsection{Representation of Objects}

Rule: only conversions and instances of types with known representation are legal, i.e. there cannot be a variable/result/parameter of type Any.
Note to self: There is a backdoor using type variables?
What about p = block : Block[Any]?
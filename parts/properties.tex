\section{Properties}

Properties are special non-instantiable types that can be inherited by other types but that will not be inherited by their subtypes.
Properties can be used wherever an inheritance operator is legal.
Their meaning is given by external tools.
The Language defines several properties that are used to influence the behaviour of the Tyr compiler.
Properties can be compared to attributes in C++ or annotations in Java.

\subsection{Predefined Properties}

\subsubsection{native}
States that the defined entity is implemented by the compiler.
If the defined entity is unknown to the compiler, an error will be issued and compilation will be aborted without producing a result.

\paragraph{Usage Restrictions}
Only types, classes and type members.


\subsubsection{CT}
Requires, that the value stored in the entity is a compile time constant.
For instance, an integer literal or the name of a type can be supplied.
This property is used to allow compile-time evaluation of type-level functions.

\paragraph{Usage Restrictions}
Variables, parameters, values.


\subsubsection{covariant}
$\texttt{type T(V : Type <: covariant)} \Rightarrow \forall u,v \in \texttt{Type}. u <: v \Rightarrow T(u) <: T(v)$.
This means that results of a type-level function use an inheritance hierarchy that equals their arguments.

\paragraph{Usage Restrictions}
Type parameters.


\subsubsection{contravariant}

$\texttt{type T(V : Type <: contravariant)} \Rightarrow \forall u,v \in \texttt{Type}. u <: v \Rightarrow T(v) <: T(u)$.
This means that results of a type-level function use an inheritance hierarchy that equals the opposite of their arguments.

\paragraph{Usage Restrictions}
Type parameters.


\subsubsection{generic}
States that only one implementation should be used to represent all values for a given type variable, i.e. the type variable is only used for type checking.
The representation will use the super type of the type variable.

\paragraph{Usage Restrictions}
Type parameters with a super type with known representation.

\subsubsection{public}
\subsubsection{protected}
\subsubsection{private}

\section{Syntax}

The syntax of Tyr is inspired by Scala and SKilL.

\subsection{Keywords, Operators and Identifiers}

Keywords in Tyr are:
\begin{center}
class
	\quad do
    \quad def
	\quad else
	\quad for
	\quad if
	\quad return
	\quad type
    \quad val
	\quad var
	\quad while
	\quad with
\end{center}
Operator Keywords are:
\begin{center}
 :
\quad <:
\quad :=
\quad =
\end{center}

Keywords can in no context be used as identifiers or operators.
They can be made available by using literal identifiers.

Operators and identifiers can name function declarations that are treated differently when it comes to their invocation.
There are no other differences.
Literal identifiers containing only operator symbols can be matched by equivalent operator invocations in an expression.

\subsection{Literals}


\begin{figure}
	\begin{grammar}
		<int> ::= ('0'-'9')+
		
		<hex> ::= ('0'-'0'|'A'-'F'|'a'-'f')+
		
		<Integer> ::= '-'? <int> ('i' <int>?)?
		
		<HexInteger> ::= "0x" <hex> ('i' <int>?)?
		
		<long> ::= '-'? <int> 'L'
		
		<Float> ::= '-'? <int>? '.' <int> (('e'|'E') '-'? <int>)? ('f' <int>?)?
		
		<string> ::= '\verb|"|' \textasciitilde['\verb|"|']* '\verb|"|'
		
		<Identifier> ::= '\verb|`|' \textasciitilde['\verb|`|']+ '\verb|`|'
		
		<Operator> ::= ???
	\end{grammar}
	\caption{Literals}
	\label{fig:syn:literals}
\end{figure}

\subsection{Grammar}

\subsubsection{Top Level Structure}

\begin{figure}
	\begin{grammar}
		<file> ::= <scopeImport>* <typeDeclaration>+
		
		<scopeImport> ::= "with" <simpleExpression> ';'?
		
		<typeDeclaration> ::= "type" <staticTypeDeclaration>
		\alt "class" <classDeclaration>
		\alt "interface" <interfaceDeclaration>
		\alt "property" <propertyDeclaration>
		
		<staticTypeDeclaration> ::= <Identifier> <typeExtension>? <typeBody>
		
		<classDeclaration> ::= <Identifier> <typeExtension>? <typeBody>
		
		<interfaceDeclaration> ::= <Identifier> <typeExtension>? <typeBody>
		
		<propertyDeclaration> ::= <Identifier> <typeBody>
	\end{grammar}
	\caption{Top Level Structure}
	\label{fig:syn:top:level}
\end{figure}

\todo{property mit type body wirkt irgendwie komisch; sollte es nicht eher ohne body, dafür aber mit parametrn/extension sein? sollten properties nicht funktional/tupel sein?}

\subsubsection{Members}

\subsubsection{Expressions}

- es muss simpleExpressions geben, die für typen und einfache ausdrücke verwendet wird
- es muss literalExpressions geben, die ausdrücke ohne berechnungen enthalten, wie etwa Typnamen; literalExpressions enthalten keine operatoren, sind also vor dem OperatorBinding-Schritt auswertbar
- statements sind expressions, aber keine simpleExpressions
- blöcke sind in diesem sinne wie statements
- d.h. wird eine expression erwartet kann man einen block als ({...}) unterbringen, nicht aber als {...}.

\subsection{Examples}

\begin{lstlisting}[language=tyr]
def foreach (p : BlockParameter[any], b : Block[void]) {
  while(move()) {
	p = get();
	b();
  }
}
def forall (p : BlockParameter[any], b : Block[bool]) : bool = {
  while(move()) {
    p = get();
    if(!b())
      return false;
  }
  return true;
}

test "usage" {
  foreach x do {
    print(x)
  }
  println(forall x do { x != null; })
}
\end{lstlisting}

Compilation of Tyr source code is done on a per-module basis.
A module is a set ouf source files with a common name.
Modules can be compared to JARs in Java or libraries in C and other languages.

\section{Module Specification}

A module must specify a name, a source directory and a set of modules it depends on.
The name of the module must not collide with any transitive dependency.
The source directory contains all sources that will be considered.
The implicit scopes created by subdirectories are relative to the source directory.
The dependencies must not depend transitively on this module.
There is an implicit dependency to \texttt{tyr.lang} for all modules except \texttt{tyr.lang}.

The name of a module creates corresponding scopes.
I.e. a definition inside the source directory of \texttt{tyr.lang} is a member of the scope \texttt{tyr.lang}.
Module names, like scope names, are all lower case.
The naming convention for modules is \textit{<organization>}.\textit{<project>}.
Examples are \texttt{tyr.lang}, \texttt{tyr.collection} or \texttt{skill.common}.
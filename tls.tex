\documentclass[a4paper,10pt]{article}

\usepackage{xltxtra}
\usepackage{polyglossia}

\usepackage{amsmath}
\usepackage{amssymb}
\usepackage[linkbordercolor={1 1 0}]{hyperref}
\usepackage[marginpar]{todo}
%lightning symbol
\usepackage{stmaryrd}

\usepackage[acronym,toc]{glossaries}

%nice quotes for parts
\usepackage{epigraph}

%table placement
\usepackage{float}

%used by grammar
\usepackage{syntax}
\setlength{\grammarindent}{3.8cm}


%used by code samples
\usepackage{color}
\usepackage{xcolor}
\usepackage{listings}
\definecolor{keywords}{HTML}{000088}
\definecolor{comments}{HTML}{409940}


\usepackage{subcaption}

\usepackage{caption}
\DeclareCaptionFont{white}{\color{white}}
\DeclareCaptionFormat{listing}{\colorbox{gray}{\parbox{0.983\textwidth}{#1#2#3}}}
\captionsetup[lstlisting]{format=listing,labelfont=white,textfont=white}

\usepackage{tikz}
\usetikzlibrary{arrows,decorations.pathmorphing,backgrounds,positioning,fit,mindmap,shapes.multipart}
\usepackage{pgfplots}
\usetikzlibrary{external}
\tikzexternalize[prefix=tikz/]
\tikzset{external/up to date check=diff}

\setromanfont[Mapping=tex-text]{Linux Libertine O}
% \setsansfont[Mapping=tex-text]{DejaVu Sans}
% \setmonofont[Mapping=tex-text]{DejaVu Sans Mono}

%paragraph numbering and toc availability
%\usepackage{titlesec}
%\titleformat{\paragraph}
%{\normalfont\normalsize\bfseries}{\theparagraph}{1em}{}
%\titlespacing*{\paragraph}
%{0pt}{3.25ex plus 1ex minus .2ex}{1.5ex plus .2ex}
\setcounter{secnumdepth}{4}
\setcounter{tocdepth}{2}

%funny makros we want to use
\newcommand{\den}[1]{\ensuremath{[\![#1]\!]}}
\DeclareMathOperator*{\bigConcat}{\bigcirc}

%skill language definition
\lstdefinelanguage{skill}
{morekeywords={interface,typedef,enum,include,with,extends,annotation,const,custom,view,as,auto,map,list,set,i1,i8,i16,i32,i64,v64,string,bool,f32,f64},
breakatwhitespace=true,
   breaklines=true,      
sensitive=false,
morecomment=[s]{/*}{*/},
morestring=[b]",
frameshape={nnn}{n}{n}{nyn},
}

%tyr language definition
\lstdefinelanguage{tyr}
{morekeywords={type,class,interface,property,with,var,val,def,test,if,else,for,while,do,return},
breakatwhitespace=true,
breaklines=true,      
sensitive=true,
morecomment=[s]{/*}{*/},
morecomment=[l]{//},
morestring=[b]",
frameshape={nnn}{n}{n}{nyn},
}
\lstset{emph={%  
    tagged,class,indexed%
    },emphstyle={\color{red}\bfseries\underbar},%
    keywordstyle=\color{keywords}\bfseries,%
    basicstyle=\normalfont\ttfamily,%
    commentstyle=\color{comments}\ttfamily,%
    stringstyle=\rmfamily,%
}%

\title{The Tyr Programming Language}
\author{Timm Felden}
\date{\today}

\makeglossaries
\include{glossary}

\begin{document}
\maketitle

\begin{abstract}
This document defines Tyr, a research language for type-oriented programming.
Type-oriented programming is a paradigm that extends on object-oriented programming.
In type-oriented languages, types are first order values like integers and objects.
An existing but primitive form of type orientation is the Java reflection API.
\end{abstract}

\renewcommand{\abstractname}{Acknowledgements}
\begin{abstract}
For Pony!
Inherits from Ada, C++, Java, Scala, SKilL.
\end{abstract}

\tableofcontents

% main parts of the document
\newpage
\part{Core Language}
\label{part:spec}

\todo{fix wording for scope/package/module/library}


\section{Introduction}

Type-oriented programming (TOP) is a paradigm that states that types are objects.
In consequence, it is possible to perform calculations on types like any other calculation.
As it is true for objects in object-oriented programming (OOP), types can be copied and may have mutable state.
The mutable state of a type can be bounded by static knowledge in the same way as pointer can be restricted to point to objects of a certain type.
As such, TOP implies OOP.

Tyr is a programming language created to explore this idea in practice.
Tyr as a language is a descendant of C++ and Scala.
In order to examine consequences of TOP for resource management, Tyr features manual memory management.

Type-level functions are descendants of C++ templates and Ada generics.


\subsection{On OOP and TOP}

TOP has no point without OOP for the following reason:

\begin{lstlisting}
type A;
type B;

val x = new (if(phi) A else B)();
\end{lstlisting}

What could be the type of x if not $A \sqcap B$?


\subsection{Guiding Principle in the Design of Tyr}

This section will explain several principles influencing the language design of Tyr.

\subparagraph{Do not define wrong properties.}

For instance, literals have a value and, hence, have a single type.
As for all types, the type of a literal must be a subtype of whatever the value of the literal is assigned to.
Likewise, the \texttt{null} pointer is an instance of each pointer type.
Therefore, there must be a bottom type of the pointer sublatice with \texttt{null} as its single instance.

\subparagraph{Flexibility of a language is the source of good library design.}

It is perfectly sane to write a type, that enriches strings with a postfix operator allowing their execution as a process in a terminal.
Also, as a more common example, collections should allow adding objects via \texttt{+}.


\subparagraph{It is not the duty of a language to ensure sanity or beauty of source code.}

If a programmer calls his variable \verb|`\n`| he is likely an idiot.
On the other hand, he could be writing a test for a source code manipulation program.

There is no point in preventing idiotic behaviour at the cost of flexibility.
Also, there is no objective measure of idiotic behaviour.
Hence, it should not be prevented by a programming language.


\subparagraph{No real language is type safe.}

It is the purpose of type systems to show obvious problems and to give the programmer an understanding of program behaviour.
Hence, a programmer may choose to violate the type system arbitrarily.

For instance, if a language allows programers to use the POSIX mmap related funcitons, he can violate basically any guarantee given by a language.


\subparagraph{Weak typing is for weak programmers.}

A type system is used to ensure that values have the properties a programer wants to make use of.
There is no point in deferring this check to runtime other than making weak programmers think their program is not as bad as it is.


\subparagraph{Pure languages are for poor programmers.}

Writing programs is about modifying the state of a computer.
The state of a computer is unrelated to mathematics in general.
\section{Syntax}

The syntax of Tyr is inspired by Scala and SKilL.


\subsection{Literals}


\begin{figure}
	\begin{grammar}
		<int> ::= ('0'-'9')+
		
		<hex> ::= ('0'-'0'|'A'-'F'|'a'-'f')+
		
		<Integer> ::= '-'? <int> ('i' <int>?)?
		
		<HexInteger> ::= "0x" <hex> ('i' <int>?)?
		
		<long> ::= '-'? <int> 'L'
		
		<Float> ::= '-'? <int>? '.' <int> (('e'|'E') '-'? <int>)? ('f' <int>?)?
		
		<string> ::= '\verb|"|' \textasciitilde['\verb|"|']* '\verb|"|'
		
		<Identifier> ::= '\verb|`|' \textasciitilde['\verb|`|']+ '\verb|`|'
	\end{grammar}
	\caption{Literals}
	\label{fig:syn:literals}
\end{figure}

\subsection{Grammar}

\subsubsection{Top Level Structure}

\subsubsection{Members}

\subsubsection{Expressions}
\section{Semantics}

The semantics of Tyr is loosely based on C++ and Scala.
\section{Properties}

Properties are special non-instantiable types that can be inherited by other types but that will not be inherited by their subtypes.
Properties can be used wherever an inheritance operator is legal.
Their meaning is given by external tools.
The Language defines several properties that are used to influence the behaviour of the Tyr compiler.
Properties can be compared to attributes in C++ or annotations in Java.

\subsection{Predefined Properties}

\subsubsection{native}
States that the defined entity is implemented by the compiler.
If the defined entity is unknown to the compiler, an error will be issued and compilation will be aborted without producing a result.

\paragraph{Usage Restrictions}
Only types, classes and type members.


\subsubsection{CT}
Requires, that the value stored in the entity is a compile time constant.
For instance, an integer literal or the name of a type can be supplied.
This property is used to allow compile-time evaluation of type-level functions.

\paragraph{Usage Restrictions}
Variables, parameters, values.


\subsubsection{covariant}
$\texttt{type T(V : Type <: covariant)} \Rightarrow \forall u,v \in \texttt{Type}. u <: v \Rightarrow T(u) <: T(v)$.
This means that results of a type-level function use an inheritance hierarchy that equals their arguments.

\paragraph{Usage Restrictions}
Type parameters.


\subsubsection{contravariant}

$\texttt{type T(V : Type <: contravariant)} \Rightarrow \forall u,v \in \texttt{Type}. u <: v \Rightarrow T(v) <: T(u)$.
This means that results of a type-level function use an inheritance hierarchy that equals the opposite of their arguments.

\paragraph{Usage Restrictions}
Type parameters.


\subsubsection{generic}
States that only one implementation should be used to represent all values for a given type variable, i.e. the type variable is only used for type checking.
The representation will use the super type of the type variable.

\paragraph{Usage Restrictions}
Type parameters with a super type with known representation.

\subsubsection{public}
\subsubsection{protected}
\subsubsection{private}

\section{Style Guide}

This section gives advice how code is to be written in the standard library.
It shall be used as a guide to readable Tyr code.

\subsection{Orderings}

\paragraph{Order of Inheritance}
The order of inheritance should be type/class, interfaces, properties.


\subsection{Naming}

\paragraph{Capitalization}

Fields, functions, types and properties start with small letters.
Classes, interfaces and type variables start with a capital letter.

var/val:
  fields
  type var -> type field (in vtable)

defs:
 def -> virtual
 static def -> static type (ada non-poly pointer)
 type def -> type method


Typen:
 Any (top)
 void (<: Any)
 bool
 Integer
 int
 byte
 long
 UnsignedInteger
 FloatingPoint
 float
 double
 pointer
 
class Object <: pointer
  String <: Object
  IterableOnce <: String
  Iterable <: IterableOnce
  Option <: Iterable
  Seq <: Iterable
  Array <: Iterable


\part{Compilation}

Compilation of Tyr source code is done on a per-module basis.
A module is a set ouf source files with a common name.
Modules can be compared to JARs in Java or libraries in C and other languages.

\section{Module Specification}

A module must specify a name, a source directory and a set of modules it depends on.
The name of the module must not collide with any transitive dependency.
The source directory contains all sources that will be considered.
The implicit scopes created by subdirectories are relative to the source directory.
The dependencies must not depend transitively on this module.
There is an implicit dependency to \texttt{tyr.lang} for all modules except \texttt{tyr.lang}.

The name of a module creates corresponding scopes.
I.e. a definition inside the source directory of \texttt{tyr.lang} is a member of the scope \texttt{tyr.lang}.
Module names, like scope names, are all lower case.
The naming convention for modules is \textit{<organization>}.\textit{<project>}.
Examples are \texttt{tyr.lang}, \texttt{tyr.collection} or \texttt{skill.common}.
\section{Elaboration of Compilation Steps}

In this section the order and strategy of elaboration of compilation steps will be discussed.

\subsection{Overview of Phases}

\begin{enumerate}
	\item Parsing: per file (text -> AST)
	\item Module AST: merge ASTs into a module AST
	\item Global and Member aggregation: Collect names and types of visible entities defined in global scopes and type/class/... members.
    \item File-Import Resolution: Resolve imports, check for duplicate names in a scope, resolve all-imports. (does not include targets of imports and import expressions)
	\item Name, Type, Import Expression and Operator Resolution: Bind names, infer and check types, turn operator applications into calls. \todo{Das hier ist eine bedarfsgetriebene evaluation von etwas, das in traditionellen Compilern in mehreren pässen erledigt werden würde}
    \item Removal of Unresolved expressions (i.e. check their absence?)
	\item Property checks: Abstract, Native
	\item Access Checks: Check visibility
	\item Removal of Block Parameters: Inline methods taking block parameters.
	\item Removal of Imports: Remove all entities introduced for implicit scope access via imports.
	\item Shadow Checks: Calculate warnings for shadowed entities
	\item IR generation: Create tl-file that can be passed to other compilations or the code generator.
\end{enumerate}

\subsection{Evaluation Order for Name Resolution}

Path imports require that expressions used in the path can be evaluated.
The problem with import paths and operator usage is that new types can be introduced by them and, hence, new dependencies can arise after definining an evaluation order.
\todo{irgendwie habe ich gerade nicht das gefühl, dass das aufgeht; vermutlich muss man die regel aufstellen, dass es keine abhängikeit an das aktuelle modul geben darf, das wäre aber enttäuschend unscharf}


\subsection{File-Import Resolution}

Check for duplicates can be performed without knowing the targets of an import expression, because they end with the name that the target will be refered under.
Names of declared entities are statically known anyway.


\subparagraph{Note}
 This step does not include duplicate names inside of block expressions.


\subsection{Removal of Block Parameters}
This step may introduce illegal usage of internal state.
Hence, access checks must happen before.
\section{Tyr Intermediate Representation}


\begin{lstlisting}[caption=Structure,language=skill]
/**
 * Any entity that can be the member of a scope.
 */
interface ScopeMember {
  /**
   * The parent of this member.
   */
  Scope parent;
}

/**
 * A static structure that can contain other static entities.
 */
Scope with ScopeMember {

  /* members of this scope can be computed by adding them to their parents */
  auto set<ScopeMember> members;

  /* the name of this scope */
  string name;
}

/**
 * The root scope as seen from this package.
 */
@singleton
RootScope extends Scope {
}

/**
 * The description of this package.
 */
@singleton
PackageInfo {
  /**
   * The name of the package as array of its parts.
   *
   * @note example: tyr.lang -> ["tyr", "lang"]
   */
  string[] name;

  /**
   * root scope of this package.
   */
  Scope packageRoot;
}
\end{lstlisting}


\begin{lstlisting}[caption=Types,language=skill]
/**
 * the actual type of an entity
 * @note no copy of a semantically equivalent type may exist
 */
interface Type {}

/**
 * A static type
 */
StaticType extends Scope with Type {
  StaticType super;

  Field[] fields;
  Function[] methods;
}

/* a simple type usage such as "int" */
DirectType extends Type {
  StaticType target;
}

/* a type resulting from an expression */
AnonymousType extends Type {
  
}
\end{lstlisting}


\begin{lstlisting}[caption=Members,language=skill]
@abstract
Field with ScopeMember {
  Type type;
  String name;
}

/* mutable fields */
Var extends Field {}

/* immutable fields */
Val extends Field {}

Function with ScopeMember {
}

Test with ScopeMember {
  string name;
  Block code;
}
\end{lstlisting}


\begin{lstlisting}[caption=Expressions,language=skill]
@abstract
Expression {
  Type type;
}

StaticAccess extends Expression {
  ScopeMember target;
}
\end{lstlisting}


\part{Libraries}

\section{IO}

 - Path (VFS)
 - File (cfile)
 - MappedFile (mmap)
 - Console


\section{Collections}

IterableOnce(T+ : Type)
- static def for (p, b)
- def foreach (f : LocalLambda[-> T])

Iterator(T+) <: IterableOnce(T+)
- empty()
- move() : bool
- get()
- for (p, b) = if(!empty) do {}

EquivalenceRelation(T : Type)
- equals(T, T) : bool
- hash(T) : int

MinimalEQR <: EquivalenceRelation
- equals := ==
- hash := .to(Int)

Iterable(T+) <: IterableOnce(T)

Seq(T+) <: Iterable(T)

Array(T) <: Seq(T)

ArrayBuffer(T) <: Seq(T)

StringBuffer <: Seq(String)

List(T) <: Seq(T)

LinkedList(T) <: List(T)

Set(T) <: Seq(T)

HList(T+) <: Seq(T)


HashSet(T : Type <: CT, Eq : Type(EquivalenceRelation) <: CT = MinimalEQR) <: Set

Map(K,V) <: Iterable(struct(K,V))

HashMap(K : Type <: CT, V : Type <: CT, Eq : Type(EquivalenceRelation(K)) <: CT) <: Set

\section{Threads}

- Thread
- ThreadPool
- Semaphore
- Mutex
- Barrier

\section{Native}

-C method placement
-C++ method placement?



\newpage
\part{Appendix}
%makes appendix count from A
\newcounter{myc}[part]
\renewcommand{\thesection}{\Alph{myc}}
\let\osection\section
\renewenvironment{section}{\stepcounter{myc}\osection}

\newpage
\printglossaries

\small{
    \bibliographystyle{alpha}
    \bibliography{skill}
}

\todos

\end{document}
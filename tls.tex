\documentclass[a4paper,10pt]{article}

\usepackage{xltxtra}
\usepackage{polyglossia}

\usepackage{amsmath}
\usepackage{amssymb}
\usepackage[linkbordercolor={1 1 0}]{hyperref}
\usepackage[marginpar]{todo}
%lightning symbol
\usepackage{stmaryrd}

\usepackage[acronym,toc]{glossaries}

%nice quotes for parts
\usepackage{epigraph}

%table placement
\usepackage{float}

%used by grammar
\usepackage{syntax}
\setlength{\grammarindent}{3.8cm}


%used by code samples
\usepackage{color}
\usepackage{xcolor}
\usepackage{listings}
\definecolor{keywords}{HTML}{000088}
\definecolor{comments}{HTML}{409940}


\usepackage{subcaption}

\usepackage{caption}
\DeclareCaptionFont{white}{\color{white}}
\DeclareCaptionFormat{listing}{\colorbox{gray}{\parbox{0.983\textwidth}{#1#2#3}}}
\captionsetup[lstlisting]{format=listing,labelfont=white,textfont=white}

\usepackage{tikz}
\usetikzlibrary{arrows,decorations.pathmorphing,backgrounds,positioning,fit,mindmap,shapes.multipart}
\usepackage{pgfplots}
\usetikzlibrary{external}
\tikzexternalize[prefix=tikz/]
\tikzset{external/up to date check=diff}

\setromanfont[Mapping=tex-text]{Linux Libertine O}
% \setsansfont[Mapping=tex-text]{DejaVu Sans}
% \setmonofont[Mapping=tex-text]{DejaVu Sans Mono}

%paragraph numbering and toc availability
%\usepackage{titlesec}
%\titleformat{\paragraph}
%{\normalfont\normalsize\bfseries}{\theparagraph}{1em}{}
%\titlespacing*{\paragraph}
%{0pt}{3.25ex plus 1ex minus .2ex}{1.5ex plus .2ex}
\setcounter{secnumdepth}{4}
\setcounter{tocdepth}{2}

%funny makros we want to use
\newcommand{\den}[1]{\ensuremath{[\![#1]\!]}}
\DeclareMathOperator*{\bigConcat}{\bigcirc}

%skill language definition
\lstdefinelanguage{skill}
{morekeywords={interface,typedef,enum,include,with,with,extends,annotation,const,custom,view,as,auto,map,list,set,i1,i8,i16,i32,i64,v64,string,bool,f32,f64},
breakatwhitespace=true,
   breaklines=true,      
sensitive=false,
morecomment=[s]{/*}{*/},
morestring=[b]",
frameshape={nnn}{n}{n}{nyn},
}
\lstset{emph={%  
    tagged,class,indexed%
    },emphstyle={\color{red}\bfseries\underbar},%
    keywordstyle=\color{keywords}\bfseries,%
    basicstyle=\normalfont\ttfamily,%
    commentstyle=\color{comments}\ttfamily,%
    stringstyle=\rmfamily,%
}%

\title{The Tyr Programming Language}
\author{Timm Felden}
\date{\today}

\makeglossaries
\include{glossary}

\begin{document}
\maketitle

\begin{abstract}
This document defines Tyr, a research language for type-oriented programming.
Type-oriented programming is a paradigm that extends on object-oriented programming.
In type-oriented languages, types are first order values like integers and objects.
An existing but primitive form of type orientation is the Java reflection API.
\end{abstract}

\renewcommand{\abstractname}{Acknowledgements}
\begin{abstract}
For Pony!
Inherits from Ada, C++, Java, Scala, SKilL.
\end{abstract}

\tableofcontents

% main parts of the document
\newpage
\part{Core Language}
\label{part:spec}


\section{Introduction}

Type-oriented programming (TOP) is a paradigm that states that types are objects.
In consequence, it is possible to perform calculations on types like any other calculation.
As it is true for objects in object-oriented programming (OOP), types can be copied and may have mutable state.
The mutable state of a type can be bounded by static knowledge in the same way as pointer can be restricted to point to objects of a certain type.
As such, TOP implies OOP.

Tyr is a programming language created to explore this idea in practice.
Tyr as a language is a descendant of C++ and Scala.
In order to examine consequences of TOP for resource management, Tyr features manual memory management.

Type-level functions are descendants of C++ templates and Ada generics.


\subsection{On OOP and TOP}

TOP has no point without OOP for the following reason:

\begin{lstlisting}
type A;
type B;

val x = new (if(phi) A else B)();
\end{lstlisting}

What could be the type of x if not $A \sqcap B$?


\subsection{Guiding Principle in the Design of Tyr}

This section will explain several principles influencing the language design of Tyr.

\subparagraph{Do not define wrong properties.}

For instance, literals have a value and, hence, have a single type.
As for all types, the type of a literal must be a subtype of whatever the value of the literal is assigned to.
Likewise, the \texttt{null} pointer is an instance of each pointer type.
Therefore, there must be a bottom type of the pointer sublatice with \texttt{null} as its single instance.

\subparagraph{Flexibility of a language is the source of good library design.}

It is perfectly sane to write a type, that enriches strings with a postfix operator allowing their execution as a process in a terminal.
Also, as a more common example, collections should allow adding objects via \texttt{+}.


\subparagraph{It is not the duty of a language to ensure sanity or beauty of source code.}

If a programmer calls his variable \verb|`\n`| he is likely an idiot.
On the other hand, he could be writing a test for a source code manipulation program.

There is no point in preventing idiotic behaviour at the cost of flexibility.
Also, there is no objective measure of idiotic behaviour.
Hence, it should not be prevented by a programming language.


\subparagraph{No real language is type safe.}

It is the purpose of type systems to show obvious problems and to give the programmer an understanding of program behaviour.
Hence, a programmer may choose to violate the type system arbitrarily.

For instance, if a language allows programers to use the POSIX mmap related funcitons, he can violate basically any guarantee given by a language.


\subparagraph{Weak typing is for weak programmers.}

A type system is used to ensure that values have the properties a programer wants to make use of.
There is no point in deferring this check to runtime other than making weak programmers think their program is not as bad as it is.


\subparagraph{Pure languages are for poor programmers.}

Writing programs is about modifying the state of a computer.
The state of a computer is unrelated to mathematics in general.
\section{Syntax}

The syntax of Tyr is inspired by Scala and SKilL.


\subsection{Literals}


\begin{figure}
	\begin{grammar}
		<int> ::= ('0'-'9')+
		
		<hex> ::= ('0'-'0'|'A'-'F'|'a'-'f')+
		
		<Integer> ::= '-'? <int> ('i' <int>?)?
		
		<HexInteger> ::= "0x" <hex> ('i' <int>?)?
		
		<long> ::= '-'? <int> 'L'
		
		<Float> ::= '-'? <int>? '.' <int> (('e'|'E') '-'? <int>)? ('f' <int>?)?
		
		<string> ::= '\verb|"|' \textasciitilde['\verb|"|']* '\verb|"|'
		
		<Identifier> ::= '\verb|`|' \textasciitilde['\verb|`|']+ '\verb|`|'
	\end{grammar}
	\caption{Literals}
	\label{fig:syn:literals}
\end{figure}

\subsection{Grammar}

\subsubsection{Top Level Structure}

\subsubsection{Members}

\subsubsection{Expressions}
\section{Semantics}

The semantics of Tyr is loosely based on C++ and Scala.

var/val:
  fields
  type var -> type field (in vtable)

defs:
 def -> virtual
 static def -> static type (ada non-poly pointer)
 type def -> type method


Typen:
 Any (top)
 void (<: Any)
 bool
 Integer
 int
 byte
 long
 UnsignedInteger
 FloatingPoint
 float
 double
 pointer
 
class Object <: pointer
  String <: Object
  IterableOnce <: String
  Iterable <: IterableOnce
  Option <: Iterable
  Seq <: Iterable
  Array <: Iterable


\part{Compilation}

modules,
source paths,
modules scopes,
default scopes,


module naming convention:
<organization>.<project>
tyr.lang
tyr.system
tyr.collection
skill.common


\part{Libraries}

\section{IO}

 - Path (VFS)
 - File (cfile)
 - MappedFile (mmap)
 - Console


\section{Collections}

IterableOnce(T : Type)
- static def for (p, b)
- def foreach (f : LocalLambda[-> T])

Iterator <: IterableOnce
- empty()
- move() : bool
- get()
- for (p, b) = if(!empty) do {}

EquivalenceRelation(T : Type)
- equals(T, T) : bool
- hash(T) : int

MinimalEQR <: EquivalenceRelation
- equals := ==
- hash := .to(Int)

Iterable <: IterableOnce

Seq <: Iterable

Array <: Seq

ArrayBuffer <: Seq

StringBuffer <: Seq(String)

List <: Seq

LinkedList <: List

Set <: Seq

HashSet(T : Type <: CT, Eq : Type(EquivalenceRelation) <: CT = MinimalEQR) <: Set

Map <: Iterable

HashMap(K : Type <: CT, V : Type <: CT, Eq : Type(EquivalenceRelation(K)) <: CT) <: Set

\section{Threads}

- Thread
- ThreadPool
- Semaphore
- Mutex
- Barrier

\section{Native}

-C method placement
-C++ method placement?



\newpage
\part{Appendix}
%makes appendix count from A
\newcounter{myc}[part]
\renewcommand{\thesection}{\Alph{myc}}
\let\osection\section
\renewenvironment{section}{\stepcounter{myc}\osection}

\newpage
\printglossaries

\small{
    \bibliographystyle{alpha}
    \bibliography{skill}
}

\todos

\end{document}